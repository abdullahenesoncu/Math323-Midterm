\documentclass{article}

% If you're new to LaTeX, here's some short tutorials:
% https://www.overleaf.com/learn/latex/Learn_LaTeX_in_30_minutes
% https://en.wikibooks.org/wiki/LaTeX/Basics

% Formatting
\usepackage[utf8]{inputenc}
\usepackage[margin=1in]{geometry}
\usepackage[titletoc,title]{appendix}

% Math
% https://www.overleaf.com/learn/latex/Mathematical_expressions
% https://en.wikibooks.org/wiki/LaTeX/Mathematics
\usepackage{amsmath,amsfonts,amssymb,mathtools}

% Images
% https://www.overleaf.com/learn/latex/Inserting_Images
% https://en.wikibooks.org/wiki/LaTeX/Floats,_Figures_and_Captions
\usepackage{graphicx,float}

% Tables
% https://www.overleaf.com/learn/latex/Tables
% https://en.wikibooks.org/wiki/LaTeX/Tables

% Algorithms
% https://www.overleaf.com/learn/latex/algorithms
% https://en.wikibooks.org/wiki/LaTeX/Algorithms
\usepackage[ruled,vlined]{algorithm2e}
\usepackage{algorithmic}

% Code syntax highlighting
% https://www.overleaf.com/learn/latex/Code_Highlighting_with_minted
\usepackage{minted}
\usemintedstyle{borland}

% References
% https://www.overleaf.com/learn/latex/Bibliography_management_in_LaTeX
% https://en.wikibooks.org/wiki/LaTeX/Bibliography_Management
\usepackage{biblatex}
\addbibresource{references.bib}
\usepackage{amsthm,amssymb}

\newcounter{problem}
\newcounter{solution}

\newcommand\Problem{%
  \stepcounter{problem}%
  \textbf{Problem \theproblem.}~%
  \setcounter{solution}{0}%
}

\newcommand\TheSolution{%
  \textbf{\\Solution:}\\%
}

\newcommand\ASolution{%
  \stepcounter{solution}%
  \textbf{Solution \thesolution:}\\%
}

% Title content
\title{Discussion on Gaussian Integers}
\author{Abdullah Enes ÖNCÜ}
\date{December 22, 2020}

\begin{document}

\maketitle



\newtheorem{theorem}{Theorem}
\newtheorem{lemma}{Lemma}


% Abstract
\begin{abstract}
    In this paper we will mention about Gaussian Integers and their relation to prime numbers and perfect squares.
\end{abstract}

% Introduction and Overview
\section{Some background we will use}

\begin{enumerate}
\item Let $a,b \in \mathbb{Z}.$ We know that $a^2 \equiv 0 \; or \; 1 (mod\; 4)$. Then we can say that $a^2+b^2 \equiv 0, \; 1, \; or \; 2(mod\; 4)$.
\item On the other hand, we know that if $p$ is an odd prime, $p \equiv 1 \; or \; 3(mod\;4)$.
\end{enumerate}
By 1. and 2., if an odd prime is of form $4n+3$, it can not be written as sum of squares.

\section{Primes can be written as sum of two squares}

I don't want to go bottom up. I firstly mention about theorem.

\begin{theorem}
Let $p \in \mathbb{N}$ be a prime number of form $4n+1$. Then there exist $a,b\in \mathbb{Z}$ such that $a^2+b^2=p$.
\end{theorem}

\begin{proof}
Assume two things, which will be proven later:
\begin{enumerate}
    \item We can find $x\in \mathbb{Z}$ such that $x^2\equiv-1(mod\; p)$.
    \item Let $c\in\mathbb{Z}$ and $c<p$. If $cp=x^2+1$ for some $x\in{Z}$, then $p$ can be written as sum of two squares.
\end{enumerate}
\: We can find $x_1\in\mathbb{Z}$ such that $x_1^2\equiv-1(mod\; p)$ by $(1.)$. Then let $x_2=p-x_1$ and also $x_2^2\equiv-1(mod\; p)$. Say $x=\min(x_1, x_2)$. We know that $x<p/2$. 
Then for some $c\in\mathbb{Z}$, $x^2+1=cp$. Because $x<p/2$, $cp < p^2/4+1$ and this means $c<p$.
By $(2.)$, there exist $a,b\in \mathbb{Z}$ such that $a^2+b^2=p$.
\end{proof}

\begin{lemma}
If $p$ is a prime of form $4n+1$, then there exist $x\in\mathbb{Z}$ such that $x^2\equiv-1(mod\;p)$.
\end{lemma}
\begin{proof}
Let $x=(-1)(-2)(-3)...(-\frac{p-1}{2})$. There is even number of term because $\frac{4n+1-1}{2}=2n$ and so $x=\frac{p-1}{2}!$
On the other hand, basically $x=(p-1)(p-2)(p-3)...(p-\frac{p-1}{2})$. \\ 
So $x^2=\frac{p-1}{2}!(\frac{p-1}{2}+1)(\frac{p-1}{2}+2)(\frac{p-1}{2}+3)...(p-1)=(p-1)!$.\\
By Wilson's Theorem, $x^2=(p-1)! \equiv -1(mod\;p)$.
\end{proof}

Let $J[i]$ be an Euclidian Ring and contains $a+bi$ for all $a,b\in\mathbb{Z}$ and it is named as Gaussian Integers.
First of all, we will prove that it is an Euclidian Ring. It is easy to show it is a ring, so we skip it.\\
\\
Let define $d(x)=d(a+bi)=a^2+b^2$ for $x\in J[i]$ and for $a,b\in\mathbb{Z}$ such that $x=a+bi$. We easily see that if $x\neq 0$ then $d(x)\geq 1$. By properties of complex numbers, $d(xy)=d(x,y)$. For $x,y\in J[i]$ where $x,y\neq 0$, $d(x).1\leq d(x)d(y)=d(xy)$. So we should prove last part.

\begin{lemma}
Let $u,v\in J[i]$. Then there exist $t,r\in J[i]$ such that $v=tu+r$ where $d(r)<d(u)$.
\end{lemma}

\begin{proof}
$\mathbb{Z}$ is a subring of $\mathbb{J}[i]$. We firstly show it for $u\in\mathbb{Z}.$ Let $v=a+bi$, $c^{'}=a\;mod\;u$ and let $c$ as the one have minimum absolute value of $c^{'}$ and $c^{'}-u$. Similarly, $d^{'}=b\;mod\;u$ and let $d$ as the one have minimum absolute value of $d^{'}$ and $d^{'}-u$. We know that, both $c$ and $d$ have absolute values less than or equal to {u/2}. So we know that, for some $e,f\in\mathbb{Z}$, $a=eu+c$ and $b=fu+d$. Moreover, $d(r)=d(c+di)\leq2(\frac{u}{2})^2<d(u)$. Condition is satisfied for $u\in\mathbb{Z}$.
\end{proof}

We proved that $\mathbb{J}[i]$ is a Euclidian Ring. Now we go for our second assumption in Theorem 1.

\begin{lemma}
Let $p$ be a prime number and $c$ is an integer which is relatively prime to $p$. If there exist two integers $x$ and $y$ satisfying $cp=x^2+y^2$, then there exist two integers $a$ and $b$ such that $p=a^2+b^2$.
\end{lemma}

\begin{proof}
Suppose $p$ be a prime element of $\mathbb{J}[i]$. We know that, $x^2+y^2=(x+yi)(x-yi)$ and because $cp$ divides it and $(c, p)=1$, $p$ divides $x+yi$ or $x-yi$. Let $p$ divides $x+yi$. Then $x+yi=p(u+vi)$ and $p(u-vi)=x-yi$. So $p$ also divides $x-yi$. Because $cp=(x+yi)(x-yi)$, it means that $p^2|cp$ but it is not possible because $p$ and $c$ are relatively prime. Contradiction. So $p$ is not a prime element of $\mathbb{J}[i]$ and $p$ can be written as product of two non unit element in $\mathbb{J}[i]$, like $p=(A+Bi)(C+Di)$.\\\\
Next, $p=(A+Bi)(C+Di)=(AC-BD)+i(AD+BC)$ and because $AD+BC=0$, $p$ also equals to $(A-Bi)(C-Di)$.
So we can say that $p^2=(A+Bi)(A-Bi)(C+Di)(C-Di)=(A^2+B^2)(C^2+D^2)$. So because $A^2+B^2$ and $C^2+D^2$ both divides $p^2$ and both $A+Bi$ and $C+Di$ are not units, then $A^2+B^2=C^2+D^2=p$ and the lemma is proven.
\end{proof}

\section{Problems}

\Problem{}
Find all the units in $\mathbb{J}[i]$.
\TheSolution{}{}
Clearly known that, $1_{\mathbb{J}[i]}=1$. So if $x$ is a unit then there exists $b\in\mathbb{J}[i]$ such that xb = $1_{\mathbb{J}[i]}=1$. So $d(x)d(b) = d(1) = 1$ and because $d(x),d(b)>0$ and both are integers, $d(x)=d(b)=1$. There are four possible $x$'s satisfying $d(x)=1$ which are $1,-1,i,-i$ and also their inverses are exist and equal to $1,-1,-i,i$ respectively. So these four numbers are units of $\mathbb{J}[i]$.

\Problem{}
If $a+bi$ is not a unit, prove that $a^2+b^2>1$. 
\TheSolution{}
Assume that $a^2+b^2=1$. Then $(a+bi)(a-bi)=a^2+b^2=1$ and so $(a+bi)$ is a unit and it gives contradiction. Note: Doesn't care about 0.

\Problem{}
Find a greatest common divisor in $\mathbb{J}[i]$:
\begin{enumerate}[label=\alph*]
    \item gcd(3+4i, 4-3i)
    \ASolution{}
    Because 3+4i
\end{enumerate}
% Example Subsection
\subsection{Subsection Title}
This is a subsection.

% Example Subsubsection
\subsubsection{Subsubsection Title}
This is a subsubsection.

%  Theoretical Background
\section{Theoretical Background}
Add your theoretical background here. Some example text: As we learned from our textbook \cite{kutz_2013}, Fourier introduced the concept of representing a given function $f(x)$ by a trigonometric series of sines and cosines:
\begin{equation}
    f(x) = \frac{a_0}{2} + \sum_{i=1}^\infty \left(a_n\cos{nx} + b_n\sin{nx}\right) \quad x \in (-\pi,\pi].
    \label{eqn:fourierseries}
\end{equation}
You can reference numbered equations, figures, tables, algorithms, and code like this: Equation~\ref{eqn:fourierseries}, etc.

% Algorithm Implementation and Development
\section{Algorithm Implementation and Development}
Add your algorithm implementation and development here. See Algorithm~\ref{alg:example} for how to include an algorithm in your document. This is how to make an \textit{ordered} list:
\begin{enumerate}
    \item Fluffy swallowed a marble.
    \item I took Fluffy to the vet.
    \item They took an ultrasound of Fluffy's intestines.
\end{enumerate}

\begin{algorithm}
\begin{algorithmic}
    \STATE{Import data from \texttt{Testdata.mat}}
    \FOR{$j = 1:20$}
        \STATE{Extract measurement $j$ from \texttt{Undata}}
        \STATE{Do something useful}
    \ENDFOR
    \IF{$i\geq 5$} 
        \STATE{$i\gets i-1$}
    \ELSE
        \IF{$i\leq 3$}
            \STATE{$i\gets i+2$}
        \ENDIF
    \ENDIF 
\end{algorithmic}
\caption{Example Algorithm}
\label{alg:example}
\end{algorithm}

% Computational Results
\section{Computational Results}
Add your computational results here. See Table~\ref{tab:mascots} for how to include a table in your document. See Figure~\ref{fig:dubs} for how to include figures in your document.

\begin{table}
    \centering
    \begin{tabular}{rll}
    & Name & Years \\
    \hline
    1 & Frosty & 1922-1930  \\
    2 & Frosty II & 1930-1936 \\
    3 & Wasky & 1946 \\
    4 & Wasky II & 1947 \\
    5 & Ski & 1954 \\
    6 & Denali & 1958 \\
    7 & King Chinook & 1959-1968\\
    8 & Regent Denali & 1969 \\
    9 & Sundodger Denali & 1981-1992 \\
    10 & King Redoubt & 1992-1998 \\
    11 & Prince Redoubt & 1998 \\
    12 & Spirit & 1999-2008 \\
    13 & Dubs I & 2009-2018 \\
    14 & Dubs II & 2018-Present
    \end{tabular}
    \caption{UW mascots as described in \cite{washington_huskies}.}
    \label{tab:mascots}
\end{table}

% begin{figure}[tb] % t = top, b = bottom, etc.
\begin{figure}
    \centering
    \includegraphics[width=0.5\linewidth]{dubs.jpg}
    \caption{Here is a picture of Dubs \cite{webeck_2018}. Dubs did not swallow a marble.}
    \label{fig:dubs}
\end{figure}

% Summary and Conclusions
\section{Summary and Conclusions}
Add your summary and conclusions here.

% References
\printbibliography

% Appendices
\begin{appendices}

% MATLAB Functions
\section{MATLAB Functions}
Add your important MATLAB functions here with a brief implementation explanation. This is how to make an \textbf{unordered} list:
\begin{itemize}
    \item \texttt{y = linspace(x1,x2,n)} returns a row vector of \texttt{n} evenly spaced points between \texttt{x1} and \texttt{x2}. 
    \item \texttt{[X,Y] = meshgrid(x,y)} returns 2-D grid coordinates based on the coordinates contained in the vectors \texttt{x} and \texttt{y}. \text{X} is a matrix where each row is a copy of \texttt{x}, and \texttt{Y} is a matrix where each column is a copy of \texttt{y}. The grid represented by the coordinates \texttt{X} and \texttt{Y} has \texttt{length(y)} rows and \texttt{length(x)} columns.  
\end{itemize}

% MATLAB Codes
\section{MATLAB Code}
Add your MATLAB code here. This section will not be included in your page limit of six pages.

\begin{listing}[h]
\inputminted{matlab}{example.m}
\caption{Example code from external file.}
\label{listing:examplecode}
\end{listing}

\end{appendices}

\end{document}
