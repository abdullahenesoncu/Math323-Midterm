\documentclass{article}

% If you're new to LaTeX, here's some short tutorials:
% https://www.overleaf.com/learn/latex/Learn_LaTeX_in_30_minutes
% https://en.wikibooks.org/wiki/LaTeX/Basics

% Formatting
\usepackage[utf8]{inputenc}
\usepackage[margin=1in]{geometry}
\usepackage[titletoc,title]{appendix}

% Math
% https://www.overleaf.com/learn/latex/Mathematical_expressions
% https://en.wikibooks.org/wiki/LaTeX/Mathematics
\usepackage{amsmath,amsfonts,amssymb,mathtools}

% Images
% https://www.overleaf.com/learn/latex/Inserting_Images
% https://en.wikibooks.org/wiki/LaTeX/Floats,_Figures_and_Captions
\usepackage{graphicx,float}

% Tables
% https://www.overleaf.com/learn/latex/Tables
% https://en.wikibooks.org/wiki/LaTeX/Tables

% Algorithms
% https://www.overleaf.com/learn/latex/algorithms
% https://en.wikibooks.org/wiki/LaTeX/Algorithms
\usepackage[ruled,vlined]{algorithm2e}
\usepackage{algorithmic}

% Code syntax highlighting
% https://www.overleaf.com/learn/latex/Code_Highlighting_with_minted
\usepackage{minted}
\usemintedstyle{borland}

% References
% https://www.overleaf.com/learn/latex/Bibliography_management_in_LaTeX
% https://en.wikibooks.org/wiki/LaTeX/Bibliography_Management
\usepackage{biblatex}
\addbibresource{references.bib}
\usepackage{amsthm,amssymb}

\newcounter{problem}
\newcounter{solution}

\newcommand\Problem{%
  \stepcounter{problem}%
  \textbf{Problem \theproblem.}~%
  \setcounter{solution}{0}%
}

\newcommand\TheSolution{%
  \textbf{\\Solution:}\\%
}

\newcommand\ASolution{%
  \stepcounter{solution}%
  \textbf{Solution \thesolution:}\\%
}

% Title content
\title{Discussion on Gaussian Integers}
\author{Abdullah Enes ÖNCÜ}
\date{December 22, 2020}

\begin{document}

\maketitle



\newtheorem{theorem}{Theorem}
\newtheorem{lemma}{Lemma}

\DeclarePairedDelimiter\ceil{\lceil}{\rceil}
\DeclarePairedDelimiter\floor{\lfloor}{\rfloor}

% Abstract
\begin{abstract}
    In this paper we will mention about Gaussian Integers and their relation to prime numbers and perfect squares.
\end{abstract}

% Introduction and Overview
\section{Some background we will use}

\begin{enumerate}
\item Let $a,b \in \mathbb{Z}.$ We know that $a^2 \equiv 0 \; or \; 1 (mod\; 4)$. Then we can say that $a^2+b^2 \equiv 0, \; 1, \; or \; 2(mod\; 4)$.
\item On the other hand, we know that if $p$ is an odd prime, $p \equiv 1 \; or \; 3(mod\;4)$.
\end{enumerate}
By 1. and 2., if an odd prime is of form $4n+3$, it can not be written as sum of squares.

\section{Primes can be written as sum of two squares}

I don't want to go bottom up. I firstly mention about theorem.

\begin{theorem}
Let $p \in \mathbb{N}$ be a prime number of form $4n+1$. Then there exist $a,b\in \mathbb{Z}$ such that $a^2+b^2=p$.
\end{theorem}

\begin{proof}
Assume two things, which will be proven later:
\begin{enumerate}
    \item We can find $x\in \mathbb{Z}$ such that $x^2\equiv-1(mod\; p)$.
    \item Let $c\in\mathbb{Z}$ and $c<p$. If $cp=x^2+1$ for some $x\in{Z}$, then $p$ can be written as sum of two squares.
\end{enumerate}
\: We can find $x_1\in\mathbb{Z}$ such that $x_1^2\equiv-1(mod\; p)$ by $(1.)$. Then let $x_2=p-x_1$ and also $x_2^2\equiv-1(mod\; p)$. Say $x=\min(x_1, x_2)$. We know that $x<p/2$. 
Then for some $c\in\mathbb{Z}$, $x^2+1=cp$. Because $x<p/2$, $cp < p^2/4+1$ and this means $c<p$.
By $(2.)$, there exist $a,b\in \mathbb{Z}$ such that $a^2+b^2=p$.
\end{proof}

\begin{lemma}
If $p$ is a prime of form $4n+1$, then there exist $x\in\mathbb{Z}$ such that $x^2\equiv-1(mod\;p)$.
\end{lemma}
\begin{proof}
Let $x=(-1)(-2)(-3)...(-\frac{p-1}{2})$. There is even number of term because $\frac{4n+1-1}{2}=2n$ and so $x=\frac{p-1}{2}!$
On the other hand, basically $x=(p-1)(p-2)(p-3)...(p-\frac{p-1}{2})$. \\ 
So $x^2=\frac{p-1}{2}!(\frac{p-1}{2}+1)(\frac{p-1}{2}+2)(\frac{p-1}{2}+3)...(p-1)=(p-1)!$.\\
By Wilson's Theorem, $x^2=(p-1)! \equiv -1(mod\;p)$.
\end{proof}

Let $J[i]$ be an Euclidian Ring and contains $a+bi$ for all $a,b\in\mathbb{Z}$ and it is named as Gaussian Integers.
First of all, we will prove that it is an Euclidian Ring. It is easy to show it is a ring, so we skip it.\\
\\
Let define $d(x)=d(a+bi)=a^2+b^2$ for $x\in J[i]$ and for $a,b\in\mathbb{Z}$ such that $x=a+bi$. We easily see that if $x\neq 0$ then $d(x)\geq 1$. By properties of complex numbers, $d(xy)=d(x,y)$. For $x,y\in J[i]$ where $x,y\neq 0$, $d(x).1\leq d(x)d(y)=d(xy)$. So we should prove last part.

\begin{lemma}
Let $u,v\in J[i]$. Then there exist $t,r\in J[i]$ such that $v=tu+r$ where $d(r)<d(u)$.
\end{lemma}

\begin{proof}
$\mathbb{Z}$ is a subring of $\mathbb{J}[i]$. We firstly show it for $u\in\mathbb{Z}.$ Let $v=a+bi$, $c^{'}=a\;mod\;u$ and let $c$ as the one have minimum absolute value of $c^{'}$ and $c^{'}-u$. Similarly, $d^{'}=b\;mod\;u$ and let $d$ as the one have minimum absolute value of $d^{'}$ and $d^{'}-u$. We know that, both $c$ and $d$ have absolute values less than or equal to {u/2}. So we know that, for some $e,f\in\mathbb{Z}$, $a=eu+c$ and $b=fu+d$. Moreover, $d(r)=d(c+di)\leq2(\frac{u}{2})^2<d(u)$. Condition is satisfied for $u\in\mathbb{Z}$.
\end{proof}

We proved that $\mathbb{J}[i]$ is a Euclidian Ring. Now we go for our second assumption in Theorem 1.

\begin{lemma}
Let $p$ be a prime number and $c$ is an integer which is relatively prime to $p$. If there exist two integers $x$ and $y$ satisfying $cp=x^2+y^2$, then there exist two integers $a$ and $b$ such that $p=a^2+b^2$.
\end{lemma}

\begin{proof}
Suppose $p$ be a prime element of $\mathbb{J}[i]$. We know that, $x^2+y^2=(x+yi)(x-yi)$ and because $cp$ divides it and $(c, p)=1$, $p$ divides $x+yi$ or $x-yi$. Let $p$ divides $x+yi$. Then $x+yi=p(u+vi)$ and $p(u-vi)=x-yi$. So $p$ also divides $x-yi$. Because $cp=(x+yi)(x-yi)$, it means that $p^2|cp$ but it is not possible because $p$ and $c$ are relatively prime. Contradiction. So $p$ is not a prime element of $\mathbb{J}[i]$ and $p$ can be written as product of two non unit element in $\mathbb{J}[i]$, like $p=(A+Bi)(C+Di)$.\\\\
Next, $p=(A+Bi)(C+Di)=(AC-BD)+i(AD+BC)$ and because $AD+BC=0$, $p$ also equals to $(A-Bi)(C-Di)$.
So we can say that $p^2=(A+Bi)(A-Bi)(C+Di)(C-Di)=(A^2+B^2)(C^2+D^2)$. So because $A^2+B^2$ and $C^2+D^2$ both divides $p^2$ and both $A+Bi$ and $C+Di$ are not units, then $A^2+B^2=C^2+D^2=p$ and the lemma is proven.
\end{proof}

\section{Problems}

\Problem{}
\\
Find all the units in $\mathbb{J}[i]$.
\TheSolution{}{}
Clearly known that, $1_{\mathbb{J}[i]}=1$. So if $x$ is a unit then there exists $b\in\mathbb{J}[i]$ such that xb = $1_{\mathbb{J}[i]}=1$. So $d(x)d(b) = d(1) = 1$ and because $d(x),d(b)>0$ and both are integers, $d(x)=d(b)=1$. There are four possible $x$'s satisfying $d(x)=1$ which are $1,-1,i,-i$ and also their inverses are exist and equal to $1,-1,-i,i$ respectively. So these four numbers are units of $\mathbb{J}[i]$.\\



\Problem{}
If $a+bi$ is not a unit, prove that $a^2+b^2>1$. 
\TheSolution{}
Assume that $a^2+b^2=1$. Then $(a+bi)(a-bi)=a^2+b^2=1$ and so $(a+bi)$ is a unit and it gives contradiction. Note: Doesn't care about 0.\\



\Problem{}
Find a greatest common divisor in $\mathbb{J}[i]$:
\begin{enumerate}
    \item $gcd(3+4i, 4-3i)$\\
    \ASolution{}
     $3+4i$ is equal to $4-3i$ up to association. That means, $3+4i=(4-3i)i$. So their gcd is $(3+4i)$ up to association or we can say it eqauls to $(3+4i)u$ where $u$ is a unit in gaussian ring.
    \item $gcd(18-i, 11+7i)$\\
    \ASolution{}
    $d(18-i)=325$ and $d(11+7i)=170$. Let $g=gcd(18-i,11+7i)$. Then, $d(g)|gcd(325,170)=5.$ So $d(g)=5$ or $d(g)=1$. Then $g$ is equal to $3+4i$, $3-4i$, or $1$ up to association. And because $3+4i$ doesn't divide $11+7i$ and $3-4i$ doesn't divide $18-i$, $g$ is $1$ up to association.
\end{enumerate}\\


\Problem{}
Prove that if $p$ is a prime number of the form $4n + 3$, then there is no $x$ such that $x^2 \equiv -1 (mod p)$.
\TheSolution{}
Assume there exist $x$ such that $x^2 \equiv -1 (mod p)$. Then $(p-x)^2=p^2-2px+x^2 \equiv x^2 \equiv -1 (mod p)$. Let $a=min(p-x,x)$ we know that $a\leq |\frac{p}{2}|$, then $a^2+1=pk\leq |\frac{p}{2}|^2+1<p^2$. So by $Lemma\;3$, there exist $c,d\in \mathbb{Z}$ such that $c^2+d^2=p$. But we know that $c^2+d^2$ cannot be equal to $3$ in modulo $4$. Contradiction. So there is no such $x$.\\


\Problem{}
Prove that no prime of the form $4n + 3$ can be written as $a^2+b^2$ where $a$ and $b$ are integers.
\TheSolution{}
$a^2+b^2\equiv0,1,2(mod\;4)$.\\


\Problem{}
Prove that there is an infinite number of primes of the form $4n+3$.
\TheSolution{}
Assume that there are finite numbers of such prime numbers. And let the set $S=\{p_1,p_2,...,p_n\}$ be the set of them. Let $X=\prod_{i=1}^{n}p_i$ and if $X$ is of form $4n+1$, let $Q=X+2$ else let $Q=X+4$. Then we know that $Q$ is bigger than all $p_i$'s, $Q$ is of the form $4n+3$. On the other hand, the prime factorization of $Q$ is $q_1^{a_1}q_2^{a_2}...q_m^{a_m}$. No element in $S$ divides $Q$ basically, because $X$ is divided by all of them and $Q-X=2\;or\;4$ is not divided by any prime number in $S$. So all $q_i$'s are of the form $4n+1$. But then $Q$ must be of the form $4n+1$. So $Q$ also has no prime divisor of the form $4n+1$. So $Q$ is a prime of the form $4n+3$. Contradiction. So the set of prime numbers of the form $4n+3$ is not finite.\\

\Problem{}
Prove that there is an infinite number of primes of the form $4n+1$.
\TheSolution{}
Assume that finite, and $S=\{p_1,p_2,...,p_n\}$ contains all primes of the form $n+1$. Let $Q=(2p_1p_2...p_n)^2+1$. $Q$ is of the form $4n+1$ and there is not any $p_i$ dividing $Q$. We can say $Q=a^2+1$ and if $Q$ has a prime divisor $p$ then we know that it is not in $S$ so it has form $4n+3$. But related to some theorem of fermat, if $p|a^2+1$ and $p$ is odd, then $p$ is of the form $4n+1$. Contradiction. So $S$ is infinite.\\


\Problem{}
Determine all the prime elements in $\mathbb{J}[i]$.
\TheSolution{}
Let $\pi$ be a prime element of gaussian ring. Then $d(\pi)=\pi\Bar{\pi}=a$ is an integer and has at most two prime divisor. Because if not, $\pi$ would have more than one prime divisor in gaussian ring. So $a$ could be product of two primes or $a$ is a prime number. Let $a=pq$, where $p$ and $q$ are prime numbers in integers. We know that $d(\pi\Bar{\pi})=d(\pi)d(\Bar{\pi})=d(\pi)d(\pi)=d(a)=d(pq)=d(p)d(q)$. Because $p$ and $q$ are integer primes, the only way to satisfy this equation is $p=q$. \\
So there is two possibility. $d(\pi)=p^2$ or $d(\pi)=p$. If $\pi\Bar{\pi}=p^2$, $d(\pi)=d(\Bar{\pi}=p)$. So if $p$ is of the form $4n+3$, because $p$ couldn't be written as $a^2+b^2$, $p$ is a prime.\\
If $d(\pi)=p$ then two conjeguate prime divisors of $p$ are prime numbers in gaussian ring if $p$ is of the form $4n+1$ or $p=2$.\\

\Problem{}
Determine all positive integers which can be written as a sum of two squares (of integers).
\TheSolution{}
\\
\end{document}
